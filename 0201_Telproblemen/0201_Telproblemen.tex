\documentclass[12pt,twoside]{article}

\textwidth 17cm \textheight 25cm \evensidemargin 0cm
\oddsidemargin 0cm \topmargin -2.5cm
\parindent 0pt
%\parskip \bigskipamount

\usepackage{graphicx}
\usepackage[dutch]{babel}
\usepackage{amssymb,amsthm,amsmath}
\usepackage[utf8]{inputenc}
\usepackage{nopageno}
\usepackage{pdfpages}
\usepackage{enumerate}
\usepackage{caption}
\usepackage{wrapfig}
\usepackage{pgf,tikz}
\usepackage{color}
\usetikzlibrary{arrows}
\usetikzlibrary{patterns}
\usepackage{fancyhdr}
\pagestyle{fancy}
\usepackage[version=3]{mhchem}
\usepackage{multicol}
\usepackage{fix-cm}
\usepackage{setspace}
\usepackage{mhchem}
\usepackage{xhfill}
\usepackage{parskip}
\usepackage{cancel}
\usepackage{mdframed}
\usepackage{url}

\newcommand{\todo}[1]{{\color{red} TODO: #1}}

\newcommand{\degree}{\ensuremath{^\circ}}
\newcommand\rad{\qopname\relax o{\mathrm{rad}}}

\newcommand\ggd{\qopname\relax o{\mathrm{ggd}}}

\def\LRA{\Leftrightarrow}%\mkern40mu}

\newcommand{\zrmbox}{\framebox{\phantom{EXE}}\phantom{X}}
\newcommand{\zrm}[1]{\framebox{#1}}

% environment oefening:
% houdt een teller bij die de oefeningen nummert, probeert ook de oefening op één pagina te houden
\newcounter{noefening}
\setcounter{noefening}{0}
\newenvironment{oefening}
{
  \stepcounter{noefening}
  \pagebreak[0]
  \begin{minipage}{\textwidth}
  \vspace*{0.7cm}{\large\bf Oefening \arabic{noefening}}
}{%
  \end{minipage}
}

\usepackage{calc}

% vraag
\reversemarginpar
\newcounter{punten}
\setcounter{punten}{0}
\newcounter{nvraag}
\setcounter{nvraag}{1}
\newlength{\puntwidth}
\newlength{\boxwidth}
\newcommand{\vraag}[1]{
\settowidth{\puntwidth}{\Large{#1}}
\setlength{\boxwidth}{1.5cm}
\addtolength{\boxwidth}{-\puntwidth}
{\large\bf Vraag \arabic{nvraag} \addtocounter{nvraag}{1}}\vspace*{-0.5cm}
{\marginpar{\color{lightgray}\fbox{\parbox{1.5cm}{\vspace*{1cm}\hspace*{\boxwidth}{\Large{#1}}}}}
\vspace*{0.5cm}}
\addtocounter{punten}{#1}}

% arulefill
\def\arulefill{\leavevmode{\xrfill[-5pt]{0.3pt}[lightgray]\endgraf}\vspace*{0.2cm}}

% \arules{n}
\newcommand{\arules}[1]{
\color{lightgray}
%\vspace*{0.05cm}
\foreach \n in {1,...,#1}{
  \vspace*{0.75cm}
  \hrule height 0.3pt\hfill
}\color{black}\vspace*{0.2cm}}

% \arule{x}
\newcommand{\arule}[1]{
\color{lightgray}{\raisebox{-0.1cm}{\rule[-0.05cm]{#1}{0.3pt}}}\color{black}
}

% \abox{y}
\newcommand{\abox}[1]{
\fbox{
\begin{minipage}{\textwidth- 4\fboxsep}
\hspace*{\textwidth}\vspace{#1}
\end{minipage}
}
}

\newcommand{\ruitjes}[1]{
\definecolor{cqcqcq}{rgb}{0.85,0.85,0.85}
\hspace*{-2.5cm}
\begin{tikzpicture}[scale=1.04,line cap=round,line join=round,>=triangle 45,x=1.0cm,y=1.0cm]
\draw [color=cqcqcq, xstep=0.5cm, ystep=0.5cm] (0,-#1) grid (20.5,0);
\end{tikzpicture}
}


\newcommand{\assenstelsel}[5][1]{
\definecolor{cqcqcq}{rgb}{0.65,0.65,0.65}
\begin{tikzpicture}[scale=#1,line cap=round,line join=round,>=triangle 45,x=1.0cm,y=1.0cm]
\draw [color=cqcqcq,dash pattern=on 1pt off 1pt, xstep=1.0cm,ystep=1.0cm] (#2,#4) grid (#3,#5);
\draw[->,color=black] (#2,0) -- (#3,0);
\draw[shift={(1,0)},color=black] (0pt,2pt) -- (0pt,-2pt) node[below] {\footnotesize $1$};
\draw[color=black] (#3.25,0.07) node [anchor=south west] { x};
\draw[->,color=black] (0,#4) -- (0,#5);
\draw[shift={(0,1)},color=black] (2pt,0pt) -- (-2pt,0pt) node[left] {\footnotesize $1$};
\draw[color=black] (0.09,#5.25) node [anchor=west] { y};
\draw[color=black] (0pt,-10pt) node[right] {\footnotesize $0$};
\end{tikzpicture}
}

\newcommand{\getallenas}[3][1]{
\definecolor{cqcqcq}{rgb}{0.65,0.65,0.65}
\begin{tikzpicture}[scale=#1,line cap=round,line join=round,>=triangle 45,x=1.0cm,y=1.0cm]
\draw [color=cqcqcq,dash pattern=on 1pt off 1pt, xstep=1.0cm,ystep=1.0cm] (#2,-0.2) grid (#3,0.2);
\draw[->,color=black] (#2.25,0) -- (#3.5,0);
\draw[shift={(0,0)},color=black] (0pt,2pt) -- (0pt,-2pt) node[below] {\footnotesize $0$};
\draw[shift={(1,0)},color=black] (0pt,2pt) -- (0pt,-2pt) node[below] {\footnotesize $1$};
\draw[color=black] (#3.25,0.07) node [anchor=south west] {$\mathbb{R}$};
\end{tikzpicture}
}

\newcommand{\visgraad}[1]{\begin{tabular}{p{0.5cm}|p{#1}}&\\\hline\\\end{tabular}}

\newcommand{\tekenschema}[2]{\begin{tabular}{p{0.5cm}|p{#1}}&\\\hline\\[#2]\end{tabular}}

% schema van Horner
\newcommand{\schemahorner}{
\begin{tabular}{p{0.5cm}|p{7cm}}
&\\[1.5cm]
\hline\\
\end{tabular}}

% geef tabular iets meer ruimte
\setlength{\tabcolsep}{14pt}
\renewcommand{\arraystretch}{1.5}

\newcommand{\toets}[3]{
\thispagestyle{plain}
\vspace*{-2.5cm}
\begin{tikzpicture}[remember picture, overlay]
    \node [shift={(15.25 cm,-1.6cm)}] {%
        \includegraphics[width=1.8cm]{/home/ppareit/kaa1415/logokaavelgem.png}%
    };%
\end{tikzpicture}

\begin{tabular}{|llc|c|}
\hline
\vspace*{-0.5cm}
&&&\\
Naam & \arule{4cm} & {\Large\bf KA AVELGEM} & \\
\vspace*{-0.75cm}
&&&\\
Klas & \arule{4cm} & {\Large\bf 20...-...-...} & \\
\hline
\vspace*{-0.75cm}
&&&\\
Toets & {\bf #2} & {\large\bf #1} & Beoordeling\\
\vspace*{-0.75cm}
&&&\\
Onderwerp & \multicolumn{2}{l|}{\bf #3} &\\
\hline
\end{tabular}
}

\newcommand{\oefeningen}[1]{

\fancyhead[LE, RO]{\vspace{0.5cm} #1}
%\thispagestyle{plain}

{\bf \Large \centering Oefeningen: #1}

}

\raggedbottom

\newcommand\dom{\qopname\relax o{\mathrm{dom}}}
\newcommand\ber{\qopname\relax o{\mathrm{ber}}}

\newcommand\mC{\qopname\relax o{\mathrm{mC}}}
\newcommand\uC{\qopname\relax o{\mathrm{{\mu}C}}}
\newcommand\C{\qopname\relax o{\mathrm{C}}}

\newcommand\W{\qopname\relax o{\mathrm{W}}}
\newcommand\kW{\qopname\relax o{\mathrm{kW}}}
\newcommand\kWh{\qopname\relax o{\mathrm{kWh}}}


\newcommand\V{\qopname\relax o{\mathrm{V}}}
\newcommand\ohm{\qopname\relax o{\mathrm{\Omega}}}
\newcommand\kohm{\qopname\relax o{\mathrm{k\Omega}}}


\newcommand\N{\qopname\relax o{\mathrm{N}}}

\newcommand\Nperkg{\qopname\relax o{\mathrm{N/kg}}}

\newcommand\Nperm{\qopname\relax o{\mathrm{N/m}}}

\newcommand\gpermol{\qopname\relax o{\mathrm{g/mol}}}


\newcommand\kgperm{\qopname\relax o{\mathrm{kg/m}}}
\newcommand\kgperdm{\qopname\relax o{\mathrm{kg/dm}}}
\newcommand\gpercm{\qopname\relax o{\mathrm{g/cm}}}
\newcommand\gperml{\qopname\relax o{\mathrm{g/ml}}}


\newcommand{\mA}{\;\mbox{mA}}
\newcommand{\A}{\;\mbox{A}}
\newcommand{\MA}{\;\mbox{MA}}

\newcommand{\us}{\;\mu\mbox{s}}
\newcommand\s{\qopname\relax o{\mathrm{s}}}

\newcommand\h{\qopname\relax o{\mathrm{h}}}

\newcommand{\kmperh}{\;\mbox{km/h}}
\newcommand{\mpers}{\;\mbox{m/s}}
\newcommand{\kmpers}{\;\mbox{km/s}}

\newcommand{\mph}{\;\mbox{mph}}

\newcommand{\Hz}{\;\mbox{Hz}}

\newcommand\Gm{\qopname\relax o{\mathrm{Gm}}}
\newcommand\Mm{\qopname\relax o{\mathrm{Mm}}}
\newcommand\km{\qopname\relax o{\mathrm{km}}}
\newcommand\hm{\qopname\relax o{\mathrm{hm}}}
\newcommand\dam{\qopname\relax o{\mathrm{dam}}}
\newcommand\m{\qopname\relax o{\mathrm{m}}}
\newcommand\dm{\qopname\relax o{\mathrm{dm}}}
\newcommand\cm{\qopname\relax o{\mathrm{cm}}}
\newcommand\mm{\qopname\relax o{\mathrm{mm}}}
\newcommand\um{\qopname\relax o{\mathrm{{\mu}m}}}
\newcommand\nm{\qopname\relax o{\mathrm{nm}}}


\newcommand\Gg{\qopname\relax o{\mathrm{Gg}}}
\newcommand\Mg{\qopname\relax o{\mathrm{Mg}}}
\newcommand\kg{\qopname\relax o{\mathrm{kg}}}
\newcommand\hg{\qopname\relax o{\mathrm{hg}}}
\renewcommand\dag{\qopname\relax o{\mathrm{dag}}}
\newcommand\g{\qopname\relax o{\mathrm{g}}}
\newcommand\dg{\qopname\relax o{\mathrm{dg}}}
\newcommand\cg{\qopname\relax o{\mathrm{cg}}}
\newcommand\mg{\qopname\relax o{\mathrm{mg}}}
\newcommand\ug{\qopname\relax o{\mathrm{{\mu}g}}}
\renewcommand\ng{\qopname\relax o{\mathrm{ng}}}

\newcommand\ton{\qopname\relax o{\mathrm{ton}}}

\newcommand\Gl{\qopname\relax o{\mathrm{Gl}}}
\newcommand\Ml{\qopname\relax o{\mathrm{Ml}}}
\newcommand\kl{\qopname\relax o{\mathrm{kl}}}
\newcommand\hl{\qopname\relax o{\mathrm{hl}}}
\newcommand\dal{\qopname\relax o{\mathrm{dal}}}
\renewcommand\l{\qopname\relax o{\mathrm{l}}}
\newcommand\dl{\qopname\relax o{\mathrm{dl}}}
\newcommand\cl{\qopname\relax o{\mathrm{cl}}}
\newcommand\ml{\qopname\relax o{\mathrm{ml}}}
\newcommand\ul{\qopname\relax o{\mathrm{{\mu}l}}}
\newcommand\nl{\qopname\relax o{\mathrm{nl}}}

\newcommand\MJ{\qopname\relax o{\mathrm{MJ}}}
\newcommand\kJ{\qopname\relax o{\mathrm{kJ}}}
\newcommand\J{\qopname\relax o{\mathrm{J}}}

\newcommand\T{\qopname\relax o{\mathrm{T}}}
\newcommand\uT{\qopname\relax o{\mathrm{{\mu}T}}}

\newcommand\grC{\qopname\relax o{\mathrm{{\degree}C}}}

\newcommand\K{\qopname\relax o{\mathrm{K}}}
\newcommand\calperK{\qopname\relax o{\mathrm{cal/K}}}

\newcommand\hPa{\qopname\relax o{\mathrm{hPa}}}
\newcommand\Pa{\qopname\relax o{\mathrm{Pa}}}

\newcommand\dB{\qopname\relax o{\mathrm{dB}}}

\newcommand{\EE}[1]{\cdot 10^{#1}}

\onehalfspacing

%\setlength{\headsep}{0cm}

\newenvironment{exlist}[1] %
{ \begin{multicols}{#1}
  \begin{enumerate}[(a)]
    \setlength{\itemsep}{0.8em} }
{ \end{enumerate}
  \end{multicols} }




\usepackage{pgfplots}

%\renewcommand{\rmdefault}{phv} % Arial
%\renewcommand{\sfdefault}{phv} % Arial

\newcommand{\dice}[1]{
\begin{tikzpicture}[x=1em,y=1em,radius=0.1]
  \draw[rounded corners=1] (0,0) rectangle (1,1);
  \ifodd#1
    \fill (0.5,0.5) circle;
  \fi
  \ifnum#1>1
    \fill (0.2,0.2) circle;
    \fill (0.8,0.8) circle;
   \ifnum#1>3
     \fill (0.2,0.8) circle;
     \fill (0.8,0.2) circle;
    \ifnum#1>5
      \fill (0.8,0.5) circle;
      \fill (0.2,0.5) circle;
    \fi
  \fi
\fi
\end{tikzpicture}
}

\begin{document}

\thispagestyle{empty}
\begin{center}
  \begin{mdframed}
  \centering
  \fontsize{40}{60}\selectfont Telproblemen
  \end{mdframed}
  \vfill
%  \includegraphics[width=\textwidth]{anewkindofcounting}
  \includegraphics[width=\textwidth]{counting-sheep}
  \vfill
\end{center}
\subsection*{Doelstellingen}
{\singlespacing
Je kan \hfill  {\scriptsize(LP 2005-069, LI 2.1)}
\begin{itemize}
  \itemsep0em
  \item telproblemen oplossen waarbij de volgorde van de elementen van een groepering van belang is en herhaling van de elementen niet mogelijk.
  \item telproblemen oplossen waarbij de volgorde van de elementen van een groepering niet van belang is en herhaling van de elementen niet mogelijk.
\end{itemize}}
%\subsection*{Algemene vaardigheden en attitudes}
%{\singlespacing
%Je \hfill {\scriptsize(LP 2006/059, ET1, ET9, ET11)}
%\begin{itemize}
%  \itemsep0em
%  \item begrijpt en gebruikt wiskundetaal.
%  \item gebruikt kennis, inzicht en vaardigheden die je verwerft in de wiskunde bij het verkennen, vertolken en verklaren van problemen uit de realiteit.
%  \item ontwikkelt zelfregulatie met betrekking tot het verwerven en verwerken van wiskundige informatie en het oplossen van problemen
%\end{itemize}
%}

\thispagestyle{empty}
\mbox{}
\newpage
\clearpage
\thispagestyle{empty}
\mbox{}
\newpage
\clearpage
\pagenumbering{arabic} 

\fancyhead[RO,LE]{Telproblemen}
\fancyhead[RE,LO]{}

\section{Inleiding}

We beginnen aan een hoofdstuk die vaak combinatoriek of combinatieleer wordt genoemd. Wij zullen voornamelijk problemen bespreken waarbij we het aantal mogelijkheden of het aantal objecten van een verzameling moeten tellen. We spreken dan ook van {\bf telproblemen}.

\section{Productregel}

\subsubsection*{Voorbeeld}

Deze zomer ben ik eens op restaurant geweest en daar hing volgende menu:

\begin{mdframed}
\begin{itemize}
  \item Voorgerechten:
  \begin{itemize}
    \item Garnaalkroketten
    \item Gerookte zalm
  \end{itemize}
  \item Hoofdgerechten:
  \begin{itemize}
    \item Lasagne
    \item Pasta met kip
    \item Biefstuk met peper-roomsaus
  \end{itemize}
  \item desserts:
  \begin{itemize}
    \item Appeltaart
    \item Sorbet
  \end{itemize}
\end{itemize}
\end{mdframed}

Ik had uiteraard veel honger en bestelde één voorgerecht, één hoofdgerecht en één dessert.

We wensen dus een maaltijd samen te stellen, we nemen dus een {\bf samengestelde beslissing} gebaseerd op verschillende onafhankelijke deelbeslissingen in een willekeurige volgorde. Logisch is dat we eerst een voorgerecht kiezen, dan een hoofdgerecht en als laatste een dessert. Onze menu zal echter niet wijzigen als we eerst een beslissing nemen over het dessert en daarna over het voorgerecht en het hoofdgerecht.

De vraag is nu:\\
{\em Op hoeveel manieren kan ik mijn menu samenstellen?}

Om zeker te zijn dat we de telling juist doen zullen we het probleem grafisch gaan voorstellen. We kunnen dit op drie manieren doen, met een boomdiagram, met een wegendiagram of met een vaasmodel. Elke voorstellingswijze heeft voor- en nadelen.

\subsection{Boomdiagram}

\hspace*{2cm} Voorgerecht \hfill Hoofdgerecht \hfill Dessert \hspace*{2cm}

\vspace*{8cm}
start
\vspace*{8cm}

In een wiskundige boom worden de lijntjes de {\bf takken} genoemd. De takken komen steeds samen in de {\bf knopen}. De knoop waar begonnen wordt noemen we de {\bf wortel} en het eindpunten van de boom worden de {\bf bladen} genoemd.

\begin{oefening}
Hoeveel mogelijke menu's kunnen we kiezen en wat hebben we moeten tellen op de boom?
\arules{2}
\end{oefening}

\begin{oefening}
Neem eens de beslissingen in een andere volgorde, namelijk kies eerst een dessert, dan een voorgerecht en dan pas het hoofdgerecht. Teken de boom. Krijg je dezelfde boom? Krijg hetzelfde aantal mogelijkheden?
\end{oefening}

\subsection{Wegendiagram}

Een boomdiagram heeft als voornaamste nadeel dat deze heel snel heel groot wordt. Als je nu eens kijkt naar de boom, dan merk je dat er per beslissing evenveel takken vertrekken uit een knoop. Laat je de vertrekkende takken dus toekomen in éénzelfde knoop dan wordt de figuur veel compacter:

\hspace*{2cm} Voorgerecht \hfill Hoofdgerecht \hfill Dessert \hspace*{2cm}

\vspace*{3cm}
start
\vspace*{3cm}

Het tellen wordt nu moeilijker, we bekijken de verschillende deelbeslissingen apart.
\begin{itemize}
  \item Voor je voorgerecht neem je een éérste beslissing, je hebt \arule{2cm} keuzes.
  \item Voor je hoofdgerecht neem je een tweede beslissing, je hebt \arule{2cm} keuzes.
  \item Voor je dessert neem je een derde beslissing, je hebt \arule{2cm} keuzes.
\end{itemize}

We weten nu dat het aantal mogelijke menu's via de methode van het wegendiagram evenveel moet zijn als via de methode van het boomdiagram.

\begin{oefening}
Hoeveel mogelijke menu's kunnen we kiezen en hoe berekenen we dit?
\arules{2}
\end{oefening}

\subsection{Vaasmodel}

We kunnen het probleem ook voorstellen met behulp van vazen. Voor elke beslissing die we moeten nemen, nemen we een vaas met in deze vaas evenveel balletjes als het aantal keuzen die deze beslissing heeft.

\hspace*{2cm} Voorgerecht \hfill Hoofdgerecht \hfill Dessert \hspace*{2cm}

\vspace*{4cm}

\begin{oefening}
Neem nu uit elke vaas een balletje, welk gerecht heb je samengesteld? Hoeveel andere gerechten kan je nog samenstellen?
\arules{2}
\end{oefening}

\subsection{Productregel}

Om een ander voorbeeld te bekijken, we wensen ons aan te kleden en we hebben keuze uit 3 truien en 2 broeken, dan kunnen we ons op $3\cdot 2 = 6$ manieren kleden. Je kiest eerst een trui wat op 3 manieren kan, en bij elk van de drie keuzes van truien heb je twee mogelijkheden om een broek te kiezen. Dit soort telproblemen kunnen we oplossen met wat we de productregel noemen:

\paragraph*{Productregel}
\begin{mdframed}
Als een telprobleem bestaat uit $r$ onafhankelijke deelbeslissingen waarbij er voor de deelbeslissingen respectievelijk $n_1, n_2, \ldots, n_r$ mogelijkheden zijn, dan zijn er in het totaal
$$n_1\cdot n_2\cdot \ldots \cdot n_r$$
mogelijkheden.
\end{mdframed}

\subsection{Oefeningen}

\begin{oefening}
In een sportclub stelt men T-shirts, shorts en kousen ter beschikking voor de spelers. Er zijn oranje, witte, blauwe, rode en groene T-shirts. De shorts zijn zwart, blauw of wit. De kousen zijn wit met een blauwe boord ofwel effen wit. Er zijn dus heel wat outfits voor de spelers mogelijk.

Teken het boomdiagram en het wegendiagram dat met de gegevens correspondeert. Hoeveel mogelijke outfits zijn er voor de spelers?
\end{oefening}

\begin{oefening}
In de kantine van een fabriek staat een machine voor warme dranken. De leverancier wil gewone koffie, espresso en thee aanbieden, met of zonder melk, zonder, met weinig of met veel suiker. Hij heeft voor elke mogelijkheid een knop voorzien.

Teken het wegendiagram. Bepaal het aantal knoppen op de machine.
\end{oefening}

\begin{oefening}
Vanaf de ingang van de zoo zijn er drie paden naar het dolfinarium. Van het dolfinarium naar de apen zijn er 4 paden. Van de apen naar de pinguïns zijn er twee paden. En van de pinguïns naar de tijgers zijn er drie paden.

Ik wil naar de show van de dolfijnen gaan kijken. Als ik van de ingang naar de dolfijnen wandel zie ik de uren van de shows geafficheerd staan en merk ik dat ik nog ruim voldoende tijd heb om eerst een bezoekje te brengen aan de apen, de pinguïns en de tijgers.

\begin{enumerate}[(a)]
  \item Hoeveel verschillende wandelingen zijn er mogelijk van de ingang tot de tijgers?
  \item Bij de tijgers pauzeer ik even tot het tijd is om terug te keren naar het dolfinarium voor de show. Ik kies een wandeling terug via de pinguïns en de apen, maar wil een pad dat ik op de heenweg nam geen tweede keer bewandelen.
  
  Schets en wegendiagram van deze tweede wandeling. Hoeveel wandelingen zijn er nu mogelijk van de tijgers naar de dolfijnen?
\end{enumerate}

\begin{center}
  \includegraphics[width=\textwidth]{zoodieren}
\end{center}
\end{oefening}

\begin{oefening}
Los volgende problemen op met behulp van een boomdiagram.
\begin{enumerate}[(a)]
  \item Hoeveel natuurlijke getallen zijn er met drie cijfers, te kiezen uit 2, 3, 5 en 7?
  \item Hoeveel natuurlijke getallen zijn er met drie {\em verschillende} cijfers, te kiezen uit 2, 3, 5 en 7?
  \item Hoeveel natuurlijke getallen van drie cijfers, te kiezen uit 2, 3, 5 en 7, beginnen met een 5?
  \item Hoeveel natuurlijke getallen van drie cijfers, te kiezen uit 2, 3, 5 en 7, eindigen op een 5?
\end{enumerate}
\end{oefening}

\section{Somregel}

\subsection{Somregel met dubbele telling}

\subsubsection*{Voorbeeld}

Vorige winter ben ik met een aantal vrienden uit het middelbaar iets gaan eten. We bestelden allemaal een hoofdgerecht en er werden 11 voorgerechten en 7 desserts besteld. Met vijf aten we zowel een voorgerecht als een dessert. Met hoeveel waren we?

\begin{oefening}
Maak een venndiagram van de verzameling van de voorgerecht-eters en de nagerecht-eters. Leidt hieruit het aantal af.
\vspace*{4cm}
\end{oefening}

Als je dus het aantal personen met een dessert het aantal personen met een voorgerecht optelt dan tel je de personen die een voorgerecht en een dessert hebben gegeten dubbel. Je kan dit compenseren door door éénmaal de personen die je dubbel hebt geteld er weer van af te trekken.

\paragraph*{Somregel}
\begin{mdframed}
Voor het aantal elementen van de unie van twee verzamelingen $A$ en $B$ geldt:
$$\#(A\cup B) = \#A + \#B - \#(A\cap B)$$
waarin
\begin{itemize}
  \item $\#A$ staat voor het aantal elementen van de verzameling $A$
  \item $A\cup B$ staat voor de unie (vereniging) van de verzamelingen $A$ en $B$, we kunnen dit ook omschrijven als $A$ {\em of} $B$
  \item $A\cap B$ staat voor de doorsnede van de verzamelingen $A$ en $B$, we kunnen dit ook omschrijven als $A$ {\em en} $B$
\end{itemize}
\end{mdframed}

\begin{oefening}
Bepaal met behulp van de somregel het aantal kaarten in een set van 52 kaarten die harten of koning zijn. Gebruik een set kaarten om dit te controleren. Maak ook een het bijhorende venndiagram.
\end{oefening}

\subsection{Somregel van elkaar uitsluitende mogelijkheden}

\subsubsection*{Voorbeeld}

Bepaal het aantal natuurlijke getallen van vier verschillende cijfers die deelbaar zijn door 5.

\begin{oefening}
Geef een goede voorstelling van het getal. Duid ook per cijfer het aantal mogelijkheden aan.
\arules{2}
\end{oefening}

Stel $B=$ de verzameling van alle mogelijke getallen van 4 verschillende cijfers die deelbaar zijn door 5

\begin{oefening}
Hoe herkennen we getallen die deelbaar zijn door vijf? Gebruik deze eigenschap om $B$ te splitsen in een aantal deelverzamelingen. Definieer de deelverzamelingen en teken het bijhorende venndiagram.
\arules{4}
\vspace*{4cm}
\end{oefening}

\begin{oefening}
Wat weten we over de deelverzamelingen?
\arules{1}
\end{oefening}

We zeggen dat de deelverzamelingen elkaar {\bf uitsluitende mogelijkheden} zijn. Dit wil zeggen dat we de somregel kunnen toepassen zonder dat we het aantal elementen in de doorsnede moeten uitrekenen.

\begin{oefening}
\begin{enumerate}[(a)]
  \item Bereken $\#B_1$ met behulp van de productregel.
  \arules{6}
  \item Bereken $\#B_2$ met behulp van de productregel.
  \arules{6}
  \item Bereken nu $\#B$ uit met behulp van de somregel en los het telprobleem op.
  \arules{2}
\end{enumerate}
\end{oefening}

\paragraph*{Somregel voor elkaar uitsluitende mogelijkheden}
\begin{mdframed}
Zijn de twee verzamelingen $A$ en $B$ elkaar uitsluitende mogelijkheden omdat ze geen elementen gemeenschappelijk hebben, dan geldt voor het aantal elementen in hun unie:
$$\#(A\cup B) = \#A + \#B$$
\end{mdframed}

\begin{oefening}
Een club bestaat uit 30 mannelijke en 50 vrouwelijke leden.
\begin{enumerate}[(a)]
  \item Op hoeveel manieren kan men 1 lid kiezen als voorzitter?
  \item Op hoeveel manieren kan men 1 mannelijk en 1 vrouwelijk lid kiezen als bestuurslid?
\end{enumerate}
\end{oefening}

\begin{oefening}
In een bibliotheek zijn 50 wetenschappelijke verhandelingen, 100 detectiveverhalen en 180 romans beschikbaar. Op hoeveel manieren kan men hieruit 1 boek kiezen? Op hoeveel manieren kan iemand een lijstje opstellen met 1 boek van elke soort?
\end{oefening}

\subsection{Praktisch}

\begin{mdframed}
Bij het oplossen van telproblemen tel je als volgt:
\begin{itemize}
  \item Kan je een boomdiagram maken, dan is het aantal mogelijkheden gelijk aan het aantal bladen.
  \item Kan je een wegendiagram maken, dan is het aantal mogelijkheden gelijk aan het product van de keuzes bij elke beslissing.
  \item Kan je een probleem herschrijven met behulp van {\bf EN}, dan gebruik je de productregel.
  \item Kan je een probleem herschrijven met behulp van {\bf OF}, dan gebruik je de somregel.
\end{itemize}
\end{mdframed}

\begin{minipage}{0.7\textwidth}
\begin{oefening}
We moeten de afloop van 10 voetbalwedstrijden voorspellen. Voor elke match zijn er 3 mogelijkheden: winnen, verliezen en gelijkspel. Hoeveel lijstjes met voorspellingen zijn er mogelijk? Hoe groot is het aantal als we zeker weten dat in de eerste wedstrijd de thuisploeg niet verliest?
\end{oefening}
\end{minipage}
\begin{minipage}{0.25\textwidth}
\hfill
  \includegraphics[width=0.6\textwidth]{voetbal}
\end{minipage}

\begin{oefening}
Voor de samenstelling van je lessenrooster moet je twee keuzevakken kiezen, elk uit een verschillend vakgebied. Er is keuze uit 3 talen, 4 wetenschappelijke vakken en 2 sporten. Hoeveel keuzemogelijkheden heb je?
\end{oefening}

\begin{oefening}
Het letterslot van een kluis bestaat uit een ring met 10 letters, een ring met 8 letters en een ring met 6 letters. Om een kluis te openen moet elke ring zo geplaatst worden dat een bepaalde stand inneemt.
\begin{enumerate}[(a)]
  \item Hoeveel lettergroepen moet een inbreker ten hoogste proberen?
  \item Hoeveel zijn er mogelijk als hij weet dat de eerste letter een E is?
\end{enumerate}
\end{oefening}

\begin{oefening}
Bij een schoolwedstrijd met 2 ploegen, die elk uit 6 deelnemers bestaan, moet elke deelnemer van de ene ploeg tegen elke deelnemer van de andere ploeg 2 partijen spelen. Hoeveel partijen zijn er in totaal?
\end{oefening}

\begin{oefening}
Bij de aankoop van een wagen moeten we kiezen tussen modellen met 2, 3, 4 of 5 deuren. De wagens met 2 of 3 deuren worden afgeleverd met een motor van 1200, 1400, 1600 of 1800 cc en de wagens met 4 of 5 deuren worden afgeleverd met een motor van 1600, 1800 of 2000 cc. Elke wagen is beschikbaar in 8 kleuren, maar voor de versies 1800 en 2000 cc zijn er bovendien nog 4 gemetalliseerde kleuren mogelijk. Voor wagens met een motor van 1200 cc bestaat er maar 1 type banden, wagens met een grotere motor kunnen naar keuze met één van 2 types banden worden uitgerust. Alle versies zijn verkrijgbaar met of zonder radio. Hoeveel mogelijkheden hebben we bij deze aankoop?
\end{oefening}

\begin{oefening}
Je hebt gewonnen op de foor. Je mag twee prijzen kiezen. Je mag kiezen uit drie verschillende tonnen. Je mag geen twee prijzen kiezen uit dezelfde ton. De eerste ton bevat 3 verschillende mp3-spelers. De tweede ton bevat 4 verschillende sleutelhangers. De laatste ton bevat 2 verschillende horloges. Hoeveel mogelijkheden om te kiezen heb je?
\end{oefening}

\begin{oefening}
Konijnenrassen bestaan in drie soorten: dwergkonijnen, gewone konijnen of reuze konijnen. Alle rassen komen voor als zwart, bruin of wit. Het gewone konijn komt ook nog voor in een gevlekte variant. Naast de rechtopstaande oren die we gewoon zijn van de konijnen komen dwerg en reuze konijnen ook voor met hangoren. Hoeveel mogelijke soorten konijnen bestaan er dan?
\end{oefening}

\begin{oefening}
Met de cijfers $1, 2, 3, \ldots , 8, 9$ worden getallen van 6 verschillende cijfers gevormd:
\begin{enumerate}[(a)]
  \item Hoeveel zulke getallen zijn er?
  \item Hoeveel van deze getallen beginnen met een 1?
  \item Hoeveel van deze getallen beginnen niet met een 5?
  \item Hoeveel van deze getallen beginnen met 55?
  \item Hoeveel van deze getallen bevatten een 4?
  \item Hoeveel van deze getallen bevatten een 3 en een 8?
  \item Hoeveel van deze getallen bevatten geen 3 en geen 8?
\end{enumerate}
\end{oefening}

\begin{oefening}
Ik ben de pincode van mijn GSM vergeten. Ik weet enkel nog dat er geen 5 in komt, maar wel een 4, en dat het allemaal verschillende cijfers waren. Hoeveel keer moet ik ten hoogste proberen om mijn code terug te vinden (in de veronderstelling dat je zoveel mag proberen als je wilt, zonder dat je sim-kaart geblokkeerd wordt)?
\end{oefening}

% \begin{oefening}
% Stel een boomdiagram op voor de natuurlijke getallen, bestaande uit drie verschillende cijfers, die je met 0, 1, 2, 3 en 4 kan vormen. Hoeveel mogelijke getallen zijn er?
% \end{oefening}
% 
% \begin{oefening}
% Hoeveel getallen, gevormd met de cijfers 0, 5, 6 en 8, liggen tussen 5000 en 6000 (deze twee uitersten inbegrepen) als een cijfer geen tweede maal mag gebruikt worden? Verduidelijk je antwoord met een boomdiagram.
% \end{oefening}
% 
% \begin{oefening}
% Je moet een menu opstellen. Hierbij moet er keuze zijn tussen 2 voorgerechten en 3 hoofdgerechten voor de {\em vegetariërs} en 3 voorgerechten en 4 hoofdgerechten voor de {\em alleseters}. Voor {\em iedereen} is er keuze uit 2 desserts. Hoeveel mogelijke menu's kan je samenstellen?
% \end{oefening}

%%%%%%%%%%%%%%%%%%%%%%%%%%%%%%%%%%%%%%%%%%%%%%%%%%%%%%%%%%%%%%%%%%%%%%
\end{document}




































