\documentclass[12pt]{article}

\textwidth 17cm \textheight 25cm \evensidemargin 0cm
\oddsidemargin 0cm \topmargin -2.5cm
\parindent 0pt
%\parskip \bigskipamount

\usepackage{graphicx}
\usepackage[dutch]{babel}
\usepackage{amssymb,amsthm,amsmath}
\usepackage[utf8]{inputenc}
\usepackage{nopageno}
\usepackage{pdfpages}
\usepackage{enumerate}
\usepackage{caption}
\usepackage{wrapfig}
\usepackage{pgf,tikz}
\usepackage{color}
\usetikzlibrary{arrows}
\usetikzlibrary{patterns}
\usepackage{fancyhdr}
\pagestyle{fancy}
\usepackage[version=3]{mhchem}
\usepackage{multicol}
\usepackage{fix-cm}
\usepackage{setspace}
\usepackage{mhchem}
\usepackage{xhfill}
\usepackage{parskip}
\usepackage{cancel}
\usepackage{mdframed}
\usepackage{url}

\newcommand{\todo}[1]{{\color{red} TODO: #1}}

\newcommand{\degree}{\ensuremath{^\circ}}
\newcommand\rad{\qopname\relax o{\mathrm{rad}}}

\newcommand\ggd{\qopname\relax o{\mathrm{ggd}}}

\def\LRA{\Leftrightarrow}%\mkern40mu}

\newcommand{\zrmbox}{\framebox{\phantom{EXE}}\phantom{X}}
\newcommand{\zrm}[1]{\framebox{#1}}

% environment oefening:
% houdt een teller bij die de oefeningen nummert, probeert ook de oefening op één pagina te houden
\newcounter{noefening}
\setcounter{noefening}{0}
\newenvironment{oefening}
{
  \stepcounter{noefening}
  \pagebreak[0]
  \begin{minipage}{\textwidth}
  \vspace*{0.7cm}{\large\bf Oefening \arabic{noefening}}
}{%
  \end{minipage}
}

\usepackage{calc}

% vraag
\reversemarginpar
\newcounter{punten}
\setcounter{punten}{0}
\newcounter{nvraag}
\setcounter{nvraag}{1}
\newlength{\puntwidth}
\newlength{\boxwidth}
\newcommand{\vraag}[1]{
\settowidth{\puntwidth}{\Large{#1}}
\setlength{\boxwidth}{1.5cm}
\addtolength{\boxwidth}{-\puntwidth}
{\large\bf Vraag \arabic{nvraag} \addtocounter{nvraag}{1}}\vspace*{-0.5cm}
{\marginpar{\color{lightgray}\fbox{\parbox{1.5cm}{\vspace*{1cm}\hspace*{\boxwidth}{\Large{#1}}}}}
\vspace*{0.5cm}}
\addtocounter{punten}{#1}}

% arulefill
\def\arulefill{\leavevmode{\xrfill[-5pt]{0.3pt}[lightgray]\endgraf}\vspace*{0.2cm}}

% \arules{n}
\newcommand{\arules}[1]{
\color{lightgray}
%\vspace*{0.05cm}
\foreach \n in {1,...,#1}{
  \vspace*{0.75cm}
  \hrule height 0.3pt\hfill
}\color{black}\vspace*{0.2cm}}

% \arule{x}
\newcommand{\arule}[1]{
\color{lightgray}{\raisebox{-0.1cm}{\rule[-0.05cm]{#1}{0.3pt}}}\color{black}
}

% \abox{y}
\newcommand{\abox}[1]{
\fbox{
\begin{minipage}{\textwidth- 4\fboxsep}
\hspace*{\textwidth}\vspace{#1}
\end{minipage}
}
}

\newcommand{\ruitjes}[1]{
\definecolor{cqcqcq}{rgb}{0.85,0.85,0.85}
\hspace*{-2.5cm}
\begin{tikzpicture}[scale=1.04,line cap=round,line join=round,>=triangle 45,x=1.0cm,y=1.0cm]
\draw [color=cqcqcq, xstep=0.5cm, ystep=0.5cm] (0,-#1) grid (20.5,0);
\end{tikzpicture}
}


\newcommand{\assenstelsel}[5][1]{
\definecolor{cqcqcq}{rgb}{0.65,0.65,0.65}
\begin{tikzpicture}[scale=#1,line cap=round,line join=round,>=triangle 45,x=1.0cm,y=1.0cm]
\draw [color=cqcqcq,dash pattern=on 1pt off 1pt, xstep=1.0cm,ystep=1.0cm] (#2,#4) grid (#3,#5);
\draw[->,color=black] (#2,0) -- (#3,0);
\draw[shift={(1,0)},color=black] (0pt,2pt) -- (0pt,-2pt) node[below] {\footnotesize $1$};
\draw[color=black] (#3.25,0.07) node [anchor=south west] { x};
\draw[->,color=black] (0,#4) -- (0,#5);
\draw[shift={(0,1)},color=black] (2pt,0pt) -- (-2pt,0pt) node[left] {\footnotesize $1$};
\draw[color=black] (0.09,#5.25) node [anchor=west] { y};
\draw[color=black] (0pt,-10pt) node[right] {\footnotesize $0$};
\end{tikzpicture}
}

\newcommand{\getallenas}[3][1]{
\definecolor{cqcqcq}{rgb}{0.65,0.65,0.65}
\begin{tikzpicture}[scale=#1,line cap=round,line join=round,>=triangle 45,x=1.0cm,y=1.0cm]
\draw [color=cqcqcq,dash pattern=on 1pt off 1pt, xstep=1.0cm,ystep=1.0cm] (#2,-0.2) grid (#3,0.2);
\draw[->,color=black] (#2.25,0) -- (#3.5,0);
\draw[shift={(0,0)},color=black] (0pt,2pt) -- (0pt,-2pt) node[below] {\footnotesize $0$};
\draw[shift={(1,0)},color=black] (0pt,2pt) -- (0pt,-2pt) node[below] {\footnotesize $1$};
\draw[color=black] (#3.25,0.07) node [anchor=south west] {$\mathbb{R}$};
\end{tikzpicture}
}

\newcommand{\visgraad}[1]{\begin{tabular}{p{0.5cm}|p{#1}}&\\\hline\\\end{tabular}}

\newcommand{\tekenschema}[2]{\begin{tabular}{p{0.5cm}|p{#1}}&\\\hline\\[#2]\end{tabular}}

% schema van Horner
\newcommand{\schemahorner}{
\begin{tabular}{p{0.5cm}|p{7cm}}
&\\[1.5cm]
\hline\\
\end{tabular}}

% geef tabular iets meer ruimte
\setlength{\tabcolsep}{14pt}
\renewcommand{\arraystretch}{1.5}

\newcommand{\toets}[3]{
\thispagestyle{plain}
\vspace*{-2.5cm}
\begin{tikzpicture}[remember picture, overlay]
    \node [shift={(15.25 cm,-1.6cm)}] {%
        \includegraphics[width=1.8cm]{/home/ppareit/kaa1415/logokaavelgem.png}%
    };%
\end{tikzpicture}

\begin{tabular}{|llc|c|}
\hline
\vspace*{-0.5cm}
&&&\\
Naam & \arule{4cm} & {\Large\bf KA AVELGEM} & \\
\vspace*{-0.75cm}
&&&\\
Klas & \arule{4cm} & {\Large\bf 20...-...-...} & \\
\hline
\vspace*{-0.75cm}
&&&\\
Toets & {\bf #2} & {\large\bf #1} & Beoordeling\\
\vspace*{-0.75cm}
&&&\\
Onderwerp & \multicolumn{2}{l|}{\bf #3} &\\
\hline
\end{tabular}
}

\newcommand{\oefeningen}[1]{

\fancyhead[LE, RO]{\vspace{0.5cm} #1}
%\thispagestyle{plain}

{\bf \Large \centering Oefeningen: #1}

}

\raggedbottom

\newcommand\dom{\qopname\relax o{\mathrm{dom}}}
\newcommand\ber{\qopname\relax o{\mathrm{ber}}}

\newcommand\mC{\qopname\relax o{\mathrm{mC}}}
\newcommand\uC{\qopname\relax o{\mathrm{{\mu}C}}}
\newcommand\C{\qopname\relax o{\mathrm{C}}}

\newcommand\W{\qopname\relax o{\mathrm{W}}}
\newcommand\kW{\qopname\relax o{\mathrm{kW}}}
\newcommand\kWh{\qopname\relax o{\mathrm{kWh}}}


\newcommand\V{\qopname\relax o{\mathrm{V}}}
\newcommand\ohm{\qopname\relax o{\mathrm{\Omega}}}
\newcommand\kohm{\qopname\relax o{\mathrm{k\Omega}}}


\newcommand\N{\qopname\relax o{\mathrm{N}}}

\newcommand\Nperkg{\qopname\relax o{\mathrm{N/kg}}}

\newcommand\Nperm{\qopname\relax o{\mathrm{N/m}}}

\newcommand\gpermol{\qopname\relax o{\mathrm{g/mol}}}


\newcommand\kgperm{\qopname\relax o{\mathrm{kg/m}}}
\newcommand\kgperdm{\qopname\relax o{\mathrm{kg/dm}}}
\newcommand\gpercm{\qopname\relax o{\mathrm{g/cm}}}
\newcommand\gperml{\qopname\relax o{\mathrm{g/ml}}}


\newcommand{\mA}{\;\mbox{mA}}
\newcommand{\A}{\;\mbox{A}}
\newcommand{\MA}{\;\mbox{MA}}

\newcommand{\us}{\;\mu\mbox{s}}
\newcommand\s{\qopname\relax o{\mathrm{s}}}

\newcommand\h{\qopname\relax o{\mathrm{h}}}

\newcommand{\kmperh}{\;\mbox{km/h}}
\newcommand{\mpers}{\;\mbox{m/s}}
\newcommand{\kmpers}{\;\mbox{km/s}}

\newcommand{\mph}{\;\mbox{mph}}

\newcommand{\Hz}{\;\mbox{Hz}}

\newcommand\Gm{\qopname\relax o{\mathrm{Gm}}}
\newcommand\Mm{\qopname\relax o{\mathrm{Mm}}}
\newcommand\km{\qopname\relax o{\mathrm{km}}}
\newcommand\hm{\qopname\relax o{\mathrm{hm}}}
\newcommand\dam{\qopname\relax o{\mathrm{dam}}}
\newcommand\m{\qopname\relax o{\mathrm{m}}}
\newcommand\dm{\qopname\relax o{\mathrm{dm}}}
\newcommand\cm{\qopname\relax o{\mathrm{cm}}}
\newcommand\mm{\qopname\relax o{\mathrm{mm}}}
\newcommand\um{\qopname\relax o{\mathrm{{\mu}m}}}
\newcommand\nm{\qopname\relax o{\mathrm{nm}}}


\newcommand\Gg{\qopname\relax o{\mathrm{Gg}}}
\newcommand\Mg{\qopname\relax o{\mathrm{Mg}}}
\newcommand\kg{\qopname\relax o{\mathrm{kg}}}
\newcommand\hg{\qopname\relax o{\mathrm{hg}}}
\renewcommand\dag{\qopname\relax o{\mathrm{dag}}}
\newcommand\g{\qopname\relax o{\mathrm{g}}}
\newcommand\dg{\qopname\relax o{\mathrm{dg}}}
\newcommand\cg{\qopname\relax o{\mathrm{cg}}}
\newcommand\mg{\qopname\relax o{\mathrm{mg}}}
\newcommand\ug{\qopname\relax o{\mathrm{{\mu}g}}}
\renewcommand\ng{\qopname\relax o{\mathrm{ng}}}

\newcommand\ton{\qopname\relax o{\mathrm{ton}}}

\newcommand\Gl{\qopname\relax o{\mathrm{Gl}}}
\newcommand\Ml{\qopname\relax o{\mathrm{Ml}}}
\newcommand\kl{\qopname\relax o{\mathrm{kl}}}
\newcommand\hl{\qopname\relax o{\mathrm{hl}}}
\newcommand\dal{\qopname\relax o{\mathrm{dal}}}
\renewcommand\l{\qopname\relax o{\mathrm{l}}}
\newcommand\dl{\qopname\relax o{\mathrm{dl}}}
\newcommand\cl{\qopname\relax o{\mathrm{cl}}}
\newcommand\ml{\qopname\relax o{\mathrm{ml}}}
\newcommand\ul{\qopname\relax o{\mathrm{{\mu}l}}}
\newcommand\nl{\qopname\relax o{\mathrm{nl}}}

\newcommand\MJ{\qopname\relax o{\mathrm{MJ}}}
\newcommand\kJ{\qopname\relax o{\mathrm{kJ}}}
\newcommand\J{\qopname\relax o{\mathrm{J}}}

\newcommand\T{\qopname\relax o{\mathrm{T}}}
\newcommand\uT{\qopname\relax o{\mathrm{{\mu}T}}}

\newcommand\grC{\qopname\relax o{\mathrm{{\degree}C}}}

\newcommand\K{\qopname\relax o{\mathrm{K}}}
\newcommand\calperK{\qopname\relax o{\mathrm{cal/K}}}

\newcommand\hPa{\qopname\relax o{\mathrm{hPa}}}
\newcommand\Pa{\qopname\relax o{\mathrm{Pa}}}

\newcommand\dB{\qopname\relax o{\mathrm{dB}}}

\newcommand{\EE}[1]{\cdot 10^{#1}}

\onehalfspacing

%\setlength{\headsep}{0cm}

\newenvironment{exlist}[1] %
{ \begin{multicols}{#1}
  \begin{enumerate}[(a)]
    \setlength{\itemsep}{0.8em} }
{ \end{enumerate}
  \end{multicols} }




\begin{document}

\thispagestyle{empty}
\begin{center}
  \begin{mdframed}
  \centering
  \fontsize{35}{70}\selectfont Veeltermfuncties
  \end{mdframed}
  \vfill
  \includegraphics[width=0.8\textwidth]{veeltermen}
  \vfill
\end{center}
%\vfill
\vspace*{-2cm}
\subsection*{Doelstellingen}
{\singlespacing

Je kan \hfill  {\scriptsize(LP 2005/069, LI 1.2, ET10,11,12,13)}
\begin{itemize}
  \item vergelijkingen van de eerste en tweede graad in één onbekende oplossen;
  \item veeltermvergelijkingen van graad hoger dan 2 oplossen met behulp van ICT;
  \item aan de hand van het functievoorschrift
        \begin{itemize}
          \item een tabel,
          \item het domein,
          \item de nulwaarden,
          \item het tekenverloop,
          \item de grafiek
        \end{itemize}
        bepalen van veeltermfuncties van de eerste en tweede graad;
  \item aan de hand van de grafiek het stijgen/dalen en de extrema van veeltermfuncties van de eerste en tweede graad bepalen;
  \item met behulp van ICT de tabel en de grafiek lezen(domein, nulwaarden, tekenverloop, stijgen/dalen, extrema) van veeltermfuncties van graad hoger dan twee;
  \item veranderingen beschrijven en vergelijken met behulp van differentiequotiënten;
  \item een vraagstuk of probleem, dat aanleiding geeft tot een veeltermfunctie, wiskundig formuleren;
  \item de door het functioneel verband bekomen vergelijking oplossen;
  \item de gevonden oplossing terug vertalen naar de oplossing van het oorspronkelijke vraagstuk of probleem;
  \item problemen met gegeven functioneel verband oplossen en deze oplossing interpreteren.
\end{itemize}

}
\thispagestyle{empty}
\mbox{}
\newpage
\clearpage
\thispagestyle{empty}
\mbox{}
\newpage
\clearpage
\pagenumbering{arabic} 


\fancyhead[RO,LE]{Veeltermfuncties}
\fancyhead[RE,LO]{}

\onehalfspacing

\section{Veeltermvergelijkingen}

\begin{oefening}
Beschouw de vergelijking
$$x^3-6x^2+5x+12=0\;.$$
\begin{enumerate}[(a)]
  \item Onderzoek of de gehele getallen van -3 tot 3 aan deze vergelijking voldoen.
  \item Zijn er nog andere getallen die aan deze vergelijking voldoen?
\end{enumerate}

\subsection{Veeltermvergelijkingen van de eerste graad}

\end{oefening}
\begin{oefening}
Los de volgende vergelijkingen van de eerste graad op in $\mathbb{R}$
\begin{enumerate}[(a)]
  \itemsep0.7em
  \item $4x-8=0$
  \item $3x+9=0$
  \item $-3x+9=0$
  \item $2(x+6)=4-(x+7)$
  \item $x - \dfrac{x-2}{3} = 4$
  \item $2(x-3)+7=5-(2-x)$
  \item $(x+1)^2-2=x(x-3)-(1-2x)$
  \item $\dfrac{3+2x}{4}-\dfrac{4x-5}{5}=\dfrac{21-6x}{6}$
  \item $\dfrac{4x}{3}-\left(\dfrac{3}{2}-\dfrac{x}{4}\right)=x+4\left(\dfrac{2x}{3}-1\right)$
  \item $\dfrac{3(x-1)}{5}-\dfrac{2(1-4x)}{7}=x+\dfrac{x+1}{5}$
  \item $\dfrac{5x}{8}-\dfrac{x-\frac{5}{2}}{4}=1$
  \item $\dfrac{x}{5}+\dfrac{x}{2}=-7$
  \item $\dfrac{3-x}{4}-\dfrac{x-2}{3}=\dfrac{x}{2}-\dfrac{4x+1}{12}$
\end{enumerate}
\end{oefening}

\pagebreak
\subsection{Veeltermvergelijkingen van de tweede graad}

\begin{oefening}
Los de volgende vergelijkingen van de tweede graad op in $\mathbb{R}$
\begin{enumerate}[(a)]
  \itemsep0.7em
  \item $2x^2-4x-6=0$
  \item $5x(x-2)=x(3x-4)-5$
  \item $\frac{2}{3}x^2+5x+1=0$
  \item $80x^2+8x=15$
  \item $80x^2+15=-8x$
  \item $(3-x)(3+x)=(x-4)^2$
  \item $(x-2)^2=x^2-4$
  \item $(3x+2)(3x-2)+6x+5=0$
  \item $x+1=(x+1)(x-1)$
  \item $4x^2-20x+25=0$
  \item $3x^2+5=0$
  \item $x=x^2$
  \item $2\left(7x+32\right)=(3x+8)^2$
  \item $11x+13=2x^2$
  \item $x\left(x+3\right)+2\left(5+2x\right)=0$
  \item $x^2+2x-8=0$
  \item $(x+3)^2=16$
  \item $9x^2+30x+25=0$
  \item $10x^2+7x-3=0$
  \item $x^2=x+1$
  \item $37x^2+13x-50=0$
  \item $3x(3x+10)+5=-20$
  \item $x^2=3x-18$
\end{enumerate}
\end{oefening}

\subsection{Veeltermvergelijkingen van graad $\geq 2$}

\begin{oefening} % www3.ul.ie/~mlc/support/.../chap3/3_3.pdf
Los volgende vergelijkingen van graad hoger dan twee op in $\mathbb{R}$:
\begin{enumerate}[(a)]
  \itemsep1em
  \item $x^3-17x^2+54x-8=0$
  \item $x^3-6x^2+11x-6=0$
  \item $x^3-7x=6$
  \item $2x^3+9x^2+7x+2=2x^2$
  \item $22x+8=3x^3+7x^2$
\end{enumerate}
\end{oefening}

\subsection{Veeltermvergelijkingen oplossen m.b.v. Geogebra}

\subsection*{Eerste methode}

\paragraph*{Voorbeeld:} Los op in $\mathbb{R}$
$$x^3-2x^2-1=x^2+x-4$$

Beschouw het linkerlid en het rechterlid als functies:
$$f(x)=x^3-2x^3-1$$
$$g(x)=x^2+x-4$$

Voor alle $x$ die een oplossing zijn van de originele vergelijk geldt dus dat
$$f(x)=g(x)$$

Construeer nu de grafieken van $f(x)$ en van $g(x)$ met Geogebra. Download het programma op \url{https://www.geogebra.org/} als dit nog niet geïnstalleerd is. Hiervoor moet je onderaan in het commandovenster het commando \verb#f(x)=x^3-2*x^3-1# en \verb#g(x)=x^2+x-4# invoeren.

Voor de snijpunten van de grafieken geldt nu dus dat $f(x)=g(x)$. En we zoeken de $x$-waarden om de originele vergelijking waar te maken. In het menu van Geogebra moeten we dus op zoek naar het icoon om het snijpunt van de twee grafieken te bepalen. Eenmaal dat we alle snijpunten hebben moeten we maar enkel de verschillende $x$-waarden van deze punten aflezen om de oplossingenverzameling te bepalen:

$$V=\{-1, 1, 3\}$$

\subsection*{Tweede methode}

\paragraph*{Voorbeeld:} Los op in $\mathbb{R}$
$$x^3-2x^2-1=x^2+x-4$$

We herschrijven deze vergelijking eerst in zijn standaardvorm:
$$\LRA x^3-3x^2-x + 3=0$$

We hoeven nu enkel het linkerlid als de functie
$$f(x)=x^3-3x^2-x + 3$$
te plotten. Want als we naar de vergelijking in standaardvorm kijken zoeken we eigenlijk de nulwaarden van deze functie.

Construeer de grafiek van deze functie $f$ in Geogebra en lees de nulwaarde af via het menu snijpunten van 2 objecten. Je zou dezelfde oplossingenverzameling

$$V=\{-1, 1, 3\}$$

moeten vinden.

\begin{oefening}
Los op in $\mathbb{R}$\\
\begin{enumerate}[(a)]
  \itemsep2em
  \item $\displaystyle\frac{{x}^{6}}{2}+\frac{21\,{x}^{5}}{4}+\frac{81\,{x}^{4}}{4}+\frac{67\,{x}^{3}}{2}+18\,{x}^{2}=\frac{27\,x}{4}+\frac{27}{4}$
  \item $\displaystyle{x}^{3}+\frac{7\,{x}^{2}}{2}+x=\frac{3}{2}$
\end{enumerate}
\end{oefening}

\begin{oefening}
Los volgende vergelijkingen op in $\mathbb{R}$ door gebruik te maken van Horner, controleer daarna je antwoord m.b.v. ICT.\\
\begin{enumerate}[(a)]
  \itemsep1em
  \item $\displaystyle{x}^{3}+{x}^{2}=2\,x$
  \item $\displaystyle{x}^{6}-{x}^{5}-13\,{x}^{4}=-13\,{x}^{3}-36\,{x}^{2}+36\,x$
\end{enumerate}
\end{oefening}

\begin{oefening}
Maak gebruik van Geogebra om de volgende vergelijkingen op te lossen in $\mathbb{R}$
\begin{enumerate}[(a)]
  \itemsep0.7em
  \item $-5x+4=0$
  \item $2(x+6)=4-(x+7)$
  \item $2x^2-4x-6=0$
  \item $5x(x-2)=x(3x-4)-5$
  \item $3x^3+x^2-8x+4=0$
  \item $x^3+6x^2-x-30=0$
  \item $x^3-3x^2+4=0$
  \item $-4(x-5)=3(2x+1)$
  \item $(x+\sqrt{5})(x^2+2x-8)=0$
  \item $5x^2+3x-4=0$
\end{enumerate}
\end{oefening}


\newpage

\section{Veeltermfuncties}

\subsection{Definities}

\paragraph*{Veeltermfunctie}
\begin{mdframed}
Een {\bf veeltermfunctie} is een reële functie waarvan het functievoorschrift gegeven wordt door een veelterm:
$$f(x)= a_nx^n + a_{n-1}x^{n-1} + \cdots + a_1x + a_0$$
\end{mdframed}

De {\bf graad van de veeltermfunctie} is gelijk aan de graad van de veelterm. In de definitie is de veeltermfunctie dus van de $n$-de graad.

\begin{oefening}
Welke van de volgende functies zijn veeltermfuncties. Als het veeltermfuncties zijn, wat is dan hun graad?
\begin{enumerate}[(a)]
  \itemsep1em
  \item $f(x)=\dfrac{2}{3}x+2$
  \item $f(x)=42$
  \item $f(x)=0.5x^2-x-1.5$
  \item $f(x)=\dfrac{3}{x^2}+\dfrac{2}{x}+1$
  \item $f(x)=-\dfrac{3}{4}x^2+3x$
  \item $f(x)=3x^2-0.5x^3$
\end{enumerate}
\end{oefening}

\subsection{Veeltermfuncties van graad 0}

\begin{itemize}
  \item Synoniemen voor deze functies zijn nulveeltermfuncties of {\bf constante functies}. Wij zullen vanaf nu de benaming constante functie gebruiken.
  \item Deze functies hebben de vorm
  $$f(x) = a$$
  met $a$ een reëel getal.
  \item De grafiek van deze functie is een horizontale rechte evenwijdig met de $x$-as.
  \item Het domein is dus $\mathbb{R}$
  \item De grafiek van deze functie snijdt de $y$-as in het punt $(0, a)$.
  \item Als $a\neq 0$ dan heeft deze functie geen nulwaarden.
  \item Als $a=0$ dan heeft deze functie alle reële getallen als nulwaarden.
\end{itemize}

\begin{oefening}
Beschouw de functie 
$$f(x)=-2\;.$$
\begin{enumerate}[(a)]
  \item Dit is een veeltermfunctie van de hoeveelste graad?
  \item Hoe noemen we zulk een functie?
  \item Hoe ziet deze functie eruit?
  \item Wat zijn de nulwaarden?
  \item Teken de grafiek zonder een functiewaardentabel te maken.
\end{enumerate}
\end{oefening}

\paragraph*{Constante functie beschrijven}
\begin{mdframed}
\begin{itemize}
  \item Functievoorschrift: $f(x)=a$
  \item $\dom f = \mathbb{R}$
  \item Nulwaarden:
  \begin{itemize}
    \item Als $a\neq 0$ dan geen nulwaarden.
    \item Als $a=0$ dan zijn de nulwaarden $\mathbb{R}$.
  \end{itemize}
  \item Tekenverloop:
  \begin{center}
    \begin{tabular}{c|lcr}
    $x$ & $-\infty$ & & $+\infty$\\
    \hline
    $f(x)$ & & teken van $a$ &       
    \end{tabular}
  \end{center}
  \item Functiewaardentabel en grafiek: /
\end{itemize}
\end{mdframed}

\begin{oefening}
Beschrijf volgende constante functies:
\begin{enumerate}[(a)]
  \item $f(x)=42$
  \item $f(x)=0$
\end{enumerate}
\end{oefening}

\subsection{Veeltermfuncties van graad 1}

\begin{itemize}
  \item Synoniemen voor deze functies zijn {\bf eerstegraadsfuncties} of lineaire functies. Wij zullen vanaf nu de benaming eerstegraadsfunctie gebruiken.
  \item Deze functies hebben de vorm
  $$f(x) = ax+b $$
  met $a$ en $b$ een reëel getallen en $a$ verschillend van $0$.
  \item De grafiek van deze functie is een schuine rechte die stijgt indien $a > 0$ en daalt indien $a<0$.
  \item Het domein is dus $\mathbb{R}$.
  \item De grafiek van deze functie snijdt de $x$-as in het punt $(-\dfrac{b}{a}, 0)$ en $x_0=-\dfrac{b}{a}$ is de enige nulwaarde van de functie.
  \item De grafiek van deze functie snijdt de $y$-as in het punt $(0, b)$.
  \item De grafiek van deze functie heeft geen extrema.
\end{itemize}

\begin{oefening}
Beschouw de functie 
$$f(x)=-\dfrac{1}{2}x+2\;.$$
\begin{enumerate}[(a)]
  \item Dit is een veeltermfunctie van de hoeveelste graad?
  \item Hoe noemen we zulk een functie?
  \item Hoe ziet deze functie eruit?
  \item Wat zijn de nulwaarden?
  \item Waar snijdt deze functie de assen?
  \item Teken de grafiek zonder een functiewaardentabel te maken.
\end{enumerate}
\end{oefening}

\paragraph*{Eerstegraadsfunctie beschrijven}
\begin{mdframed}
\begin{itemize}
  \item Functievoorschrift: $f(x)=ax + b$
  \item $\dom f = \mathbb{R}$
  \item Nulwaarde: $x_0=-\dfrac{b}{a}$
  \item Tekenverloop:
  \begin{center}
    \begin{tabular}{c|lp{2.5cm}cp{1.5cm}r}
    $x$ & $-\infty$ & & $x_0$ & & $+\infty$\\
    \hline
    $f(x)$ & & tegengesteld teken van $a$ & & teken van $a$ &       
    \end{tabular}
  \end{center}
  \item Functiewaardentabel: twee, hoogstens drie koppels is voldoende
  \item Grafiek maken
  \item Stijgen\&dalen:\\
  \begin{minipage}{0.45\textwidth}
    \centering $a>0$\\
    \begin{tabular}{c|lcr}
    $x$ & $-\infty$ & & $+\infty$\\
    \hline
    $f(x)$ & & $\nearrow$ &       
    \end{tabular}
  \end{minipage}
  \begin{minipage}{0.45\textwidth}
    \centering $a<0$\\
    \begin{tabular}{c|lcr}
    $x$ & $-\infty$ & & $+\infty$\\
    \hline
    $f(x)$ & & $\searrow$ &       
    \end{tabular}
  \end{minipage}
\end{itemize}
\end{mdframed}

\begin{oefening}
Beschrijf volgende eerstegraadsfuncties:
\begin{enumerate}[(a)]
  \item $f(x)=2x-3$
  \item $f(x)=-x+4$
\end{enumerate}
\end{oefening}

\subsection{Veeltermfuncties van graad 2}

\begin{itemize}
  \item Synoniemen voor deze functies zijn {\bf tweedegraadsfuncties} of kwadratische functies. Wij zullen vanaf nu de benaming tweedegraadsfunctie gebruiken.
  \item Deze functies hebben de vorm
  $$f(x) = ax^2+bx+c $$
  met $a$, $b$ en $c$ een reëel getallen en $a$ verschillend van $0$.
  \item We kunnen voor elke reële waarde de functiewaarde bereken, dus het domein is $\mathbb{R}$.
  \item De grafiek van deze functie is een parabool met symmetrieas evenwijdig met de $y$-as.
  \item Als $a>0$ dan opent de parabool zich naar boven en dat noemen we een {\bf dalparabool}.
  \item Als $a<0$ dan opent de parabool zich naar beneden en dat noemen we een {\bf bergparabool}.
  \item De top van de parabool bevindt zich in het punt $(-\dfrac{b}{2a}, -\dfrac{D}{4a})$, waarbij we $D=b^2-4ac$ de discriminant noemen.
  \item De top is het extrema van de functie, een maximum als $a<0$ en een minimum als $a>0$.
  \item Een tweedegraadsfunctie kan $0, 1$ of $2$ nulwaarden hebben, afhankelijk van de discriminant.
\end{itemize}

\begin{oefening}
Beschouw de functie 
$$f(x)=x^2-x-6\;.$$
\begin{enumerate}[(a)]
  \item Dit is een veeltermfunctie van de hoeveelste graad?
  \item Hoe noemen we zulk een functie?
  \item Hoe ziet deze functie eruit?
  \item Wat zijn de nulwaarden?
  \item Waar snijdt deze functie de assen?
  \item Heeft deze functie een extremum, waar ligt dit extremum en hoe noemen we dit extremum in deze context?
  \item Teken de grafiek zonder een functiewaardentabel te maken.
\end{enumerate}
\end{oefening}

\pagebreak
\subsection{Veeltermfuncties van graad 3 of hoger}

\begin{itemize}
  \item We noemen zulk een functie wel nog eens een {\bf hogeregraadsveeltermfunctie}.
  \item Deze functies hebben de algemene vorm van een veeltermfunctie
  $$f(x)= a_nx^n + a_{n-1}x^{n-1} + \cdots + a_1x + a_0\;.$$
  \item We kunnen voor elke reële waarde de functiewaarde bereken, dus het domein is $\mathbb{R}$.
  \item Een veeltermfunctie van graad $n$ heeft hoogstens $n$ nulwaarden. 
  \item Veelal vinden we de nulwaarden m.b.v. Horner.
  \item We onderzoeken best zulk een functie m.b.v. ICT, bijvoorbeeld met Geogebra.
\end{itemize}

\begin{oefening}
Met Geogebra werd de grafiek van de functie met functievoorschrift
$$f(x)=x^5-5x^3+4x$$ gegenereerd. Lees aan de hand van de grafiek het domein, de nulwaarden en de extrema af. Bepaal het tekenverloop, het stijgen en dalen en de symmetrieën van de functie.
\begin{center}
\definecolor{cqcqcq}{rgb}{0.75,0.75,0.75}
\begin{tikzpicture}[scale=0.7, line cap=round,line join=round,>=triangle 45,x=1.0cm,y=1.0cm]
\draw [color=cqcqcq,dash pattern=on 2pt off 2pt, xstep=1.0cm,ystep=1.0cm] (-7.22,-6.82) grid (8.13,6.66);
\draw[->,color=black] (-7.22,0) -- (8.13,0);
\foreach \x in {-7,-6,-5,-4,-3,-2,-1,1,2,3,4,5,6,7,8}
\draw[shift={(\x,0)},color=black] (0pt,2pt) -- (0pt,-2pt) node[below] {\footnotesize $\x$};
\draw[color=black] (7.87,0.07) node [anchor=south west] { x};
\draw[->,color=black] (0,-6.82) -- (0,6.66);
\foreach \y in {-6,-5,-4,-3,-2,-1,1,2,3,4,5,6}
\draw[shift={(0,\y)},color=black] (2pt,0pt) -- (-2pt,0pt) node[left] {\footnotesize $\y$};
\draw[color=black] (0.08,6.33) node [anchor=west] { y};
\draw[color=black] (0pt,-10pt) node[right] {\footnotesize $0$};
\clip(-7.22,-6.82) rectangle (8.13,6.66);
\draw[line width=1.6pt, smooth,samples=100,domain=-2.2:2.2] plot(\x,{(\x)^5-5*(\x)^3+4*(\x)});
\end{tikzpicture}
\end{center}
\end{oefening}

\begin{oefening}
Teken de grafiek van de eerstegraadsfunctie die als nulwaarde $x=3$ heeft en die de $y$-as snijdt in het punt $(0,-6)$. Kan je het functievoorschrift vinden voor deze functie?
\end{oefening}

\begin{oefening}
Maak voor de volgende functies een tabel en een grafiek. Noteer aan de hand van de grafiek het domein, het bereik, de nulwaarde(n), het tekenverloop, het stijgen en dalen, de extreme waarde (minima en maxima) en de symmetrieën.
\begin{enumerate}[(a)]
  \item $f(x)=\frac{2}{3}x+2$
  \item $f(x)=0.5x^2-x-1.5$
  \item $f(x)=-\frac{3}{4}x^2+3x$
  \item $f(x)=3x^2-0.5x^3$
\end{enumerate}
\end{oefening}

\begin{oefening}
Teken de grafiek van de tweedegraadsfunctie die de nulwaarden $x=-1$ en $x=3$ heeft en die als top het punt $(1,4)$ heeft. Je weet verder nog dat de grafiek de $y$-as snijdt in het punt $(0,3)$. Kan je het functievoorschrift vinden voor deze functie?
\end{oefening}

\newpage
\section{Differentiequotiënt}

\newpage
\section{Toepassingen}

\begin{oefening}
\begin{wrapfigure}[6]{r}{4cm}
\vspace*{-1.75cm}
\begin{center}
  \includegraphics[width=4cm]{CelsiusFahrenheitThermo}
\end{center}
\end{wrapfigure}
Het verband tussen het aantal graden Fahrenheit ($F$) en het aantal graden Celsius ($C$) wordt gegeven door$$F=1.8C+32\;.$$
Bij welke temperatuur $T$ is het zowel $T$ graden Celsius als $T$ graden Fahrenheit?
\end{oefening}

\begin{oefening}
Na de bereiding van een ovenschotel op een temperatuur van $180\degree C$ wordt de oven afgezet. Telkens wanneer er $5$ minuten voorbij zijn, is de temperatuur $10\degree C$ gezakt.
\begin{enumerate}[(a)]
  \item Bepaal de oventemperatuur een half uur na het uitzetten. (Tip: stel een functiewaardentabel op.)
  \item Met welke formule kun je de oventemperatuur $T$ berekenen op een willekeurig tijdstip $t$? (Maw. stel een functievoorschrift op.)
  \item Teken de grafiek. Houd er rekening mee dat, als de temperatuur tot $20\degree C$ is gedaald, het afkoelingsproces stopt.
\end{enumerate}
\end{oefening}

\begin{oefening}
Je bent lid van een comité dat een spaghetti-avond organiseert ten voordele van je sportclub. Je doel is om voor 1500 euro spaghetti-tickets te verkopen. Hoeveel moet je vragen voor een ticket voor een volwassene en hoeveel voor een ticket voor een een kind, rekening houdend met de aanwezigheid van 200 volwassenen en 100 kinderen vorig jaar.

Dit probleem heeft vele oplossingen. Je zou bijvoorbeeld 6 euro per volwassene en 3 euro per kind kunnen vragen of 4 euro per volwassene en 7 euro per kind of \ldots.
\begin{enumerate}[(a)]
  \item Noteer een functievoorschrift voor dit probleem.
  \item Het comité beslist aan alle volwassenen $5.5$ euro te vragen. Hoeveel zal dan voor elk kind gevraagd worden?
\end{enumerate}
\end{oefening}

\begin{oefening}
Een vijver heeft de vorm van een cilinder. De diameter bedraagt $2$ meter en hij is $80$ cm met water gevuld. Door verdamping daalt het waterpeil dagelijks gemiddeld met $0.3$ cm. Na hoeveel dagen is de vijver leeg?
\end{oefening}

\begin{oefening}
Je gooit een bal van het dak van de hoogste verdieping van een hoog gebouw ($60$ meter hoog). Je kunt de hoogte van de vallende bal weergeven door het voorschrift:
$$h=60-\frac{1}{2}gt^2$$
waarbij:
\begin{itemize}
  \item $h$ de hoogte uitgedrukt in meters is,
  \item $g=9.81 \;m/s^2$ de valversnelling is,
  \item $t$ het tijdstip uitgedrukt in seconden is.
\end{itemize}
\begin{enumerate}[(a)]
  \item Maak een tabel en een grafiek van de val.
  \item Na hoeveel seconden bereikt de bal de grond (rond af op een honderdste van een seconde)?
\end{enumerate}
\end{oefening}

\begin{oefening}
\begin{wrapfigure}[4]{l}{3cm}
\vspace*{-1.2cm}
\begin{center}
  \includegraphics[width=3cm]{davidscott}
\end{center}
\end{wrapfigure}

In 1971 demonstreerde astronaut Davit Scott dat op de maan een veer en een hamer even snel vallen, omdat er op de maan geen atmosfeer en dus geen luchtweerstand is. Om de maan is de valversnelling $g=1.65 \;m/s^2$.

\begin{enumerate}[(a)]
  \item Je laat op de maan een hamer en een veer vallen vanaf $5$ meter. Na hoeveel seconden bereiken ze de maanbodem (rond opnieuw af tot op een honderdste van een seconde)?
  \item Wanneer bereikt de hamer de aardbodem bij dezelfde proef?
\end{enumerate}
\end{oefening}

\begin{oefening}
Is het mogelijk dat een rechthoek een omtrek heeft van $52$ cm en een oppervlakte van $148.75$ cm$^2$? Verklaar.
\end{oefening}

\begin{oefening}
Twee schepen verlaten tegelijkertijd de haven van Zeebrugge. Het schip 'De Regenboog' vaart naar het westen en het schip 'De Viking' vaart naar het noorden. Na verloop van tijd bevinden ze zich op $270$ km afstand van elkaar. 'De Viking' heeft $50$ km meer gevaren dan 'De Regenboog'.
\begin{enumerate}[(a)]
  \item Maak een schets van de situatie.
  \item Bereken van elk schip de afgelegde afstand tot de haven van Zeebrugge (rond af op 1 $km$).
\end{enumerate}
\begin{center}
\includegraphics[width=\textwidth]{zeebrugge}
\end{center}
\end{oefening}



\end{document}
