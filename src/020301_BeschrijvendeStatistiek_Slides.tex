\documentclass[13pt]{beamer}

\usepackage[utf8]{inputenc}
\usepackage{tikz}
\usetikzlibrary{arrows,shapes}
\usepackage{graphicx}
\usepackage{gensymb}
\usepackage{verbatim}
\usepackage{multicol}
\usepackage[dutch]{babel}
\usepackage{dot2texi}

\title{Beschrijvende statistiek}
\subtitle{Leren liegen, met statistiek!}
\author{}
\institute{}
\date{}%\today}

\begin{document}

\begin{frame}
  \titlepage%
\end{frame}

\begin{frame}
  \frametitle{Beschrijvende statistiek}
  \begin{itemize}
  \item gegevens tellen
  \item kort weergeven met kengetallen
  \item grafisch voorstellen met grafieken
  \end{itemize}
  \vspace{1cm}
  \begin{description}
  \item[Onderzoeksvraag] verzamelen van gegevens
  \item[Elementen] personen of objecten die we meten
  \item[Variabele] eigenschap die we van de elementen bepalen
  \end{description}
\end{frame}

\begin{frame}[fragile]
  \frametitle{Populatie en steekproef}
  \begin{description}
  \item[populatie] verzameling {\bf alle} elementen
  \item[steekproef] deelverzameling van de populatie
    \begin{description}
    \item[representatief] alle deelgroepen vertegenwoordigd
    \item[aselect] elk element populatie evenveel kans
    \item[groot] minimum 24 elementen\\
      is de {\bf steekproefgrootte} $n$
    \end{description}
  \end{description}
  \begin{center}
    \scriptsize
    \begin{dot2tex}[dot]
      digraph G {
        graph [nodesep=0,ranksep=0];
        Variabele -> Kwalitatief;
        Variabele -> Kwantitatief;
        Kwalitatief -> Nominaal;
        Kwalitatief -> Ordinaal;
        Kwantitatief -> Discreet;
        Kwantitatief -> Continue;
      }
    \end{dot2tex}
  \end{center}

\end{frame}

\end{document}

